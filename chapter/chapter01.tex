\zihao{-4}\linespread{1.55}\selectfont
\chapter{模板使用说明}
\section{\LaTeX 介绍}
\subsection{环境配置}
高维多目标优化问题在实际应用中普遍存在,它具有3 个以上相互冲突的目标函数。
由于系统固有的不确定性、数据测量偏差和问题近似建模等,导致优化问题中往往存在
一些不确定的参数或者变量,不确定高维多目标优化问题受到了广泛的关注。在实际问
题中,不确定性参数的上、下边界的获取比概率分布函数和隶属度函数容易的多,因此
区间优化成为处理不确定性高维多目标优化问题的有效途径。虽然进化算法在直接求解
区间高维多目标优化问题具有明显的优势,但区间高维多目标进化算法的研究尚处于起
步阶段,在有效比较区间目标的优劣,获得收敛性、多样性和不确定性俱佳的Pareto 前
沿等方面存在一系列的问题和难点。为此,本文将针对区间高维多目标进化算法展开系
统的研究,主要内容如下\cite{chen2001}:

(1) 针对区间Pareto 支配在算法后期选择压力不足,提出了柔性Pareto 支配驱动
的区间高维多目标进化算法。首先在目标空间定义个体的可能最优值和不确定度,引入
柔性系数1和2 ,设计柔性区间Pareto 支配准则;然后提出个体松弛度指标去衡量个体
在目标空间的拥挤度,进一步区分相同Pareto 支配层的个体优劣。为验证算法的有效性,
与其他优秀区间算法在InDTLZ 和InWFG 测试问题上进行了比较。实验结果表明,本
文提出的算法在IIGD 指标和IX 指标方面中均优于其他算法,从而验证了该算法在扩大
候选解的支配区域、缓解选择压力、提升最优解集的多样性方面发挥了重要的作用\cite{nadkarni1992}。

(2) 针对非支配解占候选解的比例呈指数形式扩展引起的Pareto 支配失效,过分
依靠多样性维护机制容易造成Pareto 最优前沿收敛困难等问题,提出了理想超平面引导
的区间高维多目标进化算法。首先通过划分目标空间构建理想超平面,引导算法向收敛
性较强的区域搜索。然后对区间目标函数进行拉丁超立方抽样,以样本点与理想超平面
之间的平均距离衡量个体优劣,并设计随机概率选择、整体选择、平均选择等最优解选
择机制以增强种群多样性。与其它先进的区间算法相比,无论是在InDTLZ 测试问题,
还是InMaOP 测试问题,提出的算法在IIGD 指标上的最佳结果均超过总数的一半,且
在IX 指标上的表现出较强的竞争力,证明了该算法在收敛性、不确定性方面的优越性。

(3) 针对算法侧重保持某种性质并遵循 “邻域假设”,造成搜索资源的浪费,容易
陷入局部最优,难以有效平衡收敛性、多样性、以及不确定性,提出了综合评价指标选
择的区间高维多目标进化算法。首先在目标空间定义虚拟最优解,提出收敛性度量、空
间密度度量和不确定性度量准则。然后引入权重因子,通过静态参数、随机参数、动态
参数和自适应参数,设计四种综合评价指标去选择最优解集。实验仿真结果表明,在InMaOP 和InWFG 测试问题上表现最优的策略分别是自适应综合评价指标策略和静态
综合评价指标策略。此外,与其他三种优秀的算法相比,所提算法可以提供具有较好收
敛性和多样性的解决方案,且不确定性较小,从而验证了提出的综合评价指标的有效性。

(4) 针对求解具有多个特征的优化问题时,几乎不可能存在一个算法获得的结果
比其它所有的算法都要好,提出了双阶段博弈集成的区间高维多目标进化算法。首先分
析选择算子的性能差异,从选择集成角度构建了匹配选择策略池和环境选择策略池。然
后结合博弈论机制,将进化种群视为博弈玩家,设计了基于种群博弈的匹配选择算子。
最后在博弈轮选举机制的作用下,设计了基于波达计数的环境选择算子,以个体合作博
弈的方式选择最优解集。将该算法与其他六种算法在InDTLZ、InWFG 和InMaOP 测试
问题上进行比较,实验结果表明,提出的算法在三组测试问题的IIGD 指标上均具有最
多的最佳结果,从而验证了该算法能够合理利用现有算法中的策略和机制,通过多个算
子相互协调,共同作用,进一步提升了算法的整体性能。
\section{\LaTeX 介绍}
\zhlipsum[1-3]
\chapter{模板使用说明}
图 \ref{第1个子图标签名}

图 \ref{第2个子图标签名}

图 \ref{第1个图}

表 \ref{第1个表}针对求解具有多个特征的优化问题时,几乎不可能存在一个算法获得的结果
比其它所有的算法都要好,提出了双阶段博弈集成的区间高维多目标进化算法。首先分
析选择算子的性能差异,从选择集成角度构建了匹配选择策略池和环境选择策略池。然
后结合博弈论机制,将进化种群视为博弈玩家,设计了基于种群博弈的匹配选择算子。
最后在博弈轮选举机制的作用下,设计了基于波达计数的环境选择算子,以个体合作博
弈的方式选择最优解集。将该算法与其他六种算法在InDTLZ、InWFG 和InMaOP 测试
问题上进行比较,实验结果表明,提出的算法在三
\begin{figure}[htp]
	\begin{subfigure}[b]{0.3\textwidth}
		\centering
		\includegraphics[width=\textwidth]{example-image}
		\bisubcaption{中文标题}{English title}\label{第1个子图标签名}
	\end{subfigure}
	\hfill
	\begin{subfigure}[b]{0.3\textwidth}
		\centering
		\includegraphics[width=\textwidth]{example-image}
		\bisubcaption{中文标题}{English title}\label{第2个子图标签名}
	\end{subfigure}
	\hfill
	\begin{subfigure}[b]{0.3\textwidth}
		\centering
		\includegraphics[width=\textwidth]{example-image}
		\bisubcaption{中文标题}{English title}
	\end{subfigure}
	\bicaption{第一个图名字}{English title}\label{第1个图}
\end{figure}
针对求解具有多个特征的优化问题时,几乎不可能存在一个算法获得的结果
比其它所有的算法都要好,提出了双阶段博弈集成的区间高维多目标进化算法。首先分
析选择算子的性能差异,从选择集成角度构建了匹配选择策略池和环境选择策略池。然
后结合博弈论机制,将进化种群视为博弈玩家,设计了基于种群博弈的匹配选择算子。
最后在博弈轮选举机制的作用下,设计了基于波达计数的环境选择算子,以个体合作博
弈的方式选择最优解集。将该算法与其他六种算法在InDTLZ、InWFG 和InMaOP 测试
问题上进行比较,实验结果表明,提出的算法在三
针对求解具有多个特征的优化问题时,几乎不可能存在一个算法获得的结果
比其它所有的算法都要好,提出了双阶段博弈集成的区间高维多目标进化算法。首先分
析选择算子的性能差异,从选择集成角度构建了匹配选择策略池和环境选择策略池。然
后结合博弈论机制,将进化种群视为博弈玩家,设计了基于种群博弈的匹配选择算子。
最后在博弈轮选举机制的作用下,设计了基于波达计数的环境选择算子,以个体合作博
弈的方式选择最优解集。将该算法与其他六种算法在InDTLZ、InWFG 和InMaOP 测试
问题上进行比较,实验结果表明,提出的算法在三
针对求解具有多个特征的优化问题时,几乎不可能存在一个算法获得的结果
比其它所有的算法都要好,提出了双阶段博弈集成的区间高维多目标进化算法。首先分
析选择算子的性能差异,从选择集成角度构建了匹配选择策略池和环境选择策略池。然
后结合博弈论机制,将进化种群视为博弈玩家,设计了基于种群博弈的匹配选择算子。
最后在博弈轮选举机制的作用下,设计了基于波达计数的环境选择算子,以个体合作博
弈的方式选择最优解集。将该算法与其他六种算法在InDTLZ、InWFG 和InMaOP 测试
问题上进行比较,实验结果表明,提出的算法在三
针对求解具有多个特征的优化问题时,几乎不可能存在一个算法获得的结果
比其它所有的算法都要好,提出了双阶段博弈集成的区间高维多目标进化算法。首先分
析选择算子的性能差异,从选择集成角度构建了匹配选择策略池和环境选择策略池。然
后结合博弈论机制,将进化种群视为博弈玩家,设计了基于种群博弈的匹配选择算子。
最后在博弈轮选举机制的作用下,设计了基于波达计数的环境选择算子,以个体合作博
弈的方式选择最优解集。将该算法与其他六种算法在InDTLZ、InWFG 和InMaOP 测试
问题上进行比较,实验结果表明,提出的算法在三
\begin{figure}[htp]
	\centering
	\includegraphics[width=0.5\textwidth]{example-image}
	\bicaption{第二个图名字}{English title}
\end{figure}
针对求解具有多个特征的优化问题时,几乎不可能存在一个算法获得的结果
比其它所有的算法都要好,提出了双阶段博弈集成的区间高维多目标进化算法。首先分
析选择算子的性能差异,从选择集成角度构建了匹配选择策略池和环境选择策略池。然
后结合博弈论机制,将进化种群视为博弈玩家,设计了基于种群博弈的匹配选择算子。
最后在博弈轮选举机制的作用下,设计了基于波达计数的环境选择算子,以个体合作博
弈的方式选择最优解集。将该算法与其他六种算法在InDTLZ、InWFG 和InMaOP 测试
问题上进行比较,实验结果表明,提出的算法在三针对求解具有多个特征的优化问题时,几乎不可能存在一个算法获得的结果
比其它所有的算法都要好,提出了双阶段博弈集成的区间高维多目标进化算法。首先分
析选择算子的性能差异,从选择集成角度构建了匹配选择策略池和环境选择策略池。然
后结合博弈论机制,将进化种群视为博弈玩家,设计了基于种群博弈的匹配选择算子。
最后在博弈轮选举机制的作用下,设计了基于波达计数的环境选择算子,以个体合作博
弈的方式选择最优解集。将该算法与其他六种算法在InDTLZ、InWFG 和InMaOP 测试
问题上进行比较,实验结果表明,提出的算法在三
\begin{table}[H]
	\centering
	\bicaption{第一个表名字}{English title}\label{第1个表}
	\begin{tabular}{ccc}
		\toprule
		呵呵&    &   \\
		\midrule
		&    &   \\
		\midrule
		&    &   \\
		\bottomrule
	\end{tabular}
\end{table}
针对求解具有多个特征的优化问题时,几乎不可能存在一个算法获得的结果
比其它所有的算法都要好,提出了双阶段博弈集成的区间高维多目标进化算法。首先分
析选择算子的性能差异,从选择集成角度构建了匹配选择策略池和环境选择策略池。然
后结合博弈论机制,将进化种群视为博弈玩家,设计了基于种群博弈的匹配选择算子。
最后在博弈轮选举机制的作用下,设计了基于波达计数的环境选择算子,以个体合作博
弈的方式选择最优解集。将该算法与其他六种算法在InDTLZ、InWFG 和InMaOP 测试
问题上进行比较,实验结果表明,提出的算法在三
\begin{center}
\setlength{\tabcolsep}{10mm}
\begin{longtable}{ccc}
	\bicaption{第二个表名字}{English title}\label{第2个表}\\
	\endfirsthead
 	\multicolumn{3}{c}{\makecell{\zihao{5}\kaishu 表~\thetable 表格(续)\vspace{-8pt}\\\zihao{5}\kaishu Table ~\thetable 表格}}\\
	\toprule
	符号      &   表示含义              &     单位 \\
	\toprule
	\endhead
	\toprule
	符号      &   表示含义              &     单位 \\
	\toprule
	$t$      & 时间                   &     $s$\\
	\midrule
	...     &  ...                   &    ...\\
	\midrule 
	...    &  ...                   &    ...\\
	\midrule 
	...     &  ...                  &    ...\\
	\midrule
	...     &  ...                   &    ...\\
	\midrule 
	...    &  ...                   &    ...\\
	\midrule 
	...     &  ...                  &    ...\\ 
	\midrule
	...     &  ...                   &    ...\\
	\midrule 
	...    &  ...                   &    ...\\
	\midrule
	...     &  ...                   &    ...\\
	\midrule 
	...    &  ...                   &    ...\\
	\midrule 
	...     &  ...                  &    ...\\
	\midrule
	...     &  ...                   &    ...\\
	\midrule 
	...    &  ...                   &    ...\\
	\midrule 
	...     &  ...                  &    ...\\ 
	\midrule
	...     &  ...                   &    ...\\
	\midrule 
	...    &  ...                   &    ...\\
	\midrule
	...     &  ...                   &    ...\\
	\midrule 
	...    &  ...                   &    ...\\
	\midrule 
	...     &  ...                  &    ...\\
	\midrule
	...     &  ...                   &    ...\\
	\midrule 
	...    &  ...                   &    ...\\
	\midrule 
	...     &  ...                  &    ...\\ 
	\midrule
	...     &  ...                   &    ...\\
	\midrule 
	...    &  ...                   &    ...\\
	\midrule
	...     &  ...                   &    ...\\
	\midrule 
	...    &  ...                   &    ...\\
	\midrule 
	...     &  ...                  &    ...\\
	\midrule
	...     &  ...                   &    ...\\
	\midrule 
	...    &  ...                   &    ...\\
	\midrule 
	...     &  ...                  &    ...\\ 
	\midrule
	...     &  ...                   &    ...\\
	\midrule 
	...    &  ...                   &    ...\\
	\bottomrule
\end{longtable}
\end{center}
\vspace{-2em}
针对求解具有多个特征的优化问题时,几乎不可能存在一个算法获得的结果
比其它所有的算法都要好,提出了双阶段博弈集成的区间高维多目标进化算法。首先分
析选择算子的性能差异,从选择集成角度构建了匹配选择策略池和环境选择策略池。然
后结合博弈论机制,将进化种群视为博弈玩家,设计了基于种群博弈的匹配选择算子。
最后在博弈轮选举机制的作用下,设计了基于波达计数的环境选择算子,以个体合作博
弈的方式选择最优解集。将该算法与其他六种算法在InDTLZ、InWFG 和InMaOP 测试
问题上进行比较,实验结果表明,提出的算法在三针对求解具有多个特征的优化问题时,几乎不可能存在一个算法获得的结果
比其它所有的算法都要好,提出了双阶段博弈集成的区间高维多目标进化算法。首先分
析选择算子的性能差异,从选择集成角度构建了匹配选择策略池和环境选择策略池。然
后结合博弈论机制,将进化种群视为博弈玩家,设计了基于种群博弈的匹配选择算子。
最后在博弈轮选举机制的作用下,设计了基于波达计数的环境选择算子,以个体合作博
弈的方式选择最优解集。将该算法与其他六种算法在InDTLZ、InWFG 和InMaOP 测试
问题上进行比较,实验结果表明,提出的算法在三
