\begin{cabstract}

高维多目标优化问题在实际应用中普遍存在,它具有3 个以上相互冲突的目标函数。
由于系统固有的不确定性、数据测量偏差和问题近似建模等,导致优化问题中往往存在
一些不确定的参数或者变量,不确定高维多目标优化问题受到了广泛的关注。在实际问
题中,不确定性参数的上、下边界的获取比概率分布函数和隶属度函数容易的多,因此
区间优化成为处理不确定性高维多目标优化问题的有效途径。虽然进化算法在直接求解
区间高维多目标优化问题具有明显的优势,但区间高维多目标进化算法的研究尚处于起
步阶段,在有效比较区间目标的优劣,获得收敛性、多样性和不确定性俱佳的Pareto 前
沿等方面存在一系列的问题和难点。为此,本文将针对区间高维多目标进化算法展开系
统的研究,主要内容如下:
\zhlipsum[1-5]

\end{cabstract}

%\ckeywords{TeX/LaTeX, XeLaTeX与中文处理, 科技排版, 郑州大学, 学位论文
%模板, 关于摘要}
\begin{eabstract} 
%
\lipsum[1-5]
%%
%%An abstract of a dissertation is a summary and extraction of research work and
%%contributions. Included in an abstract should be description of research topic
%%and research objective, brief introduction to methodology and research
%%process, and summarization of conclusion and contributions of the research. An
%%abstract should be characterized by independence and clarity and carry
%%identical information with the dissertation. It should be such that the
%%general idea and major contributions of the dissertation are conveyed without
%%reading the dissertation. 
%%
%%
\end{eabstract}

%\ekeywords{TeX/LaTeX, XeLaTeX Chinese, Scientific typesetting system,
%Academic thesis template, Zhengzhou University, About keywords}
