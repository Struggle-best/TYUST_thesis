\section{论文无需附录去掉该部分}
嘻嘻
\section{一些测试}
\subsection{图}
\begin{figure}[H]
	\centering
	\includegraphics[scale=0.5]{example-image}
%	\caption{长标题测试。这是个很长很长很长很长很长很长很长很长很长很长很长很长很长很长很长很长很长很长很长的标题}
\end{figure}
\subsection{表格}
\begin{table}[H]
	\centering
	\caption{三线表}
	\setlength{\tabcolsep}{10mm}{
		\begin{tabular}{cc} 
			\toprule 
			输入& 输出                           \\ 
			\midrule 
			$ SB_{0} $\qquad I0.0& $ D_{1} $灯 \qquad Q0.0    \\ 
			$ SB_{1} $\qquad I0.1& $ D_{2} $灯 \qquad Q0.1    \\ 
			\bottomrule 
		\end{tabular}
	}
\end{table}
\subsection{数学公式}
\[ f(x) = \int_{-\infty}^\infty  \hat f(x)\xi\,e^{2 \pi i \xi x}  \,\mathrm{d}\xi  \]
\begin{equation} 
	\begin{bmatrix}
		z_{1}\\
		z_{2}\\
		\vdots\\
		z_{n}
	\end{bmatrix}
	=\begin{bmatrix}
		1 & x_{1}  &x_{1}^{2}  \\
		1& x_{2} & x_{2}^{2} \\
		\vdots & \vdots & \vdots\\
		1 & x_{n} & x_{3}^{2}
	\end{bmatrix}
	\begin{bmatrix}
		a_{1} \\
		a_{2} \\
		a_{3} 
	\end{bmatrix}
\end{equation}

\begin{equation}
	y_{n}=y_{max} \times e^{\left (-\frac{(x_{n}-x_{max})^{2}}{Q} \right )}
\end{equation}
\section{冒泡排序算法}
君子曰:学不可以已。青,取之于蓝,而青于蓝;冰,水为之,而寒于水。
木直中绳。(车柔)以为轮,其曲中规。虽有槁暴,不复挺者,(车柔)使之然也。故木
受绳则直, 金就砺则利,君子博学而日参省乎己,则知明而行无过矣。吾尝终日而思矣
,  不如须臾之所学也;吾尝(足齐)而望矣,不如登高之博见也。登高而招,臂非加长
也,  而见者远;  顺风而呼,  声非加疾也,而闻者彰。假舆马者,非利足也,而致千
里;假舟楫者,非能水也,而绝江河,  君子生非异也,善假于物也。积土成山,风雨兴
焉;积水成渊,蛟龙生焉;积善成德,而神明自得,圣心备焉。故不积跬步,无以至千里
;不积小流,无以成江海。骐骥一跃,不能十步;驽马十驾,功在不舍。锲而舍之,朽木
不折;  锲而不舍,金石可镂。蚓无爪牙之利,筋骨之强,上食埃土,下饮黄泉,用心一
也。蟹六跪而二螯,非蛇鳝之穴无可寄托者,用心躁也。——荀况
\begin{lstlisting}[language=Java]
	/*冒泡排序算法*/ 
	public static void bubble_sort(int[] arr) {
		int i, j, temp, len = arr.length;
		for (i = 0; i < len - 1; i++)
		for (j = 0; j < len - 1 - i; j++) 
		if (arr[j] > arr[j + 1]) {
			temp = arr[j];
			arr[j] = arr[j + 1];
			arr[j + 1] = temp;
		}
	}
\end{lstlisting}